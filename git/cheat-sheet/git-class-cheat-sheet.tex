\documentclass[footexclude,twocolumn,DIV25,fontsize=10pt]{scrreprt} 

% Author: Nate Murray (based on Steve Tayon's Clojure cheat sheet)
% Comments, errors, suggestions: nate@natemurray.com

% License
% Eclipse Public License v1.0
% http://opensource.org/licenses/eclipse-1.0.php

% Packages
\usepackage[utf8]{inputenc}
\usepackage[T1]{fontenc}
\usepackage{textcomp}
\usepackage[english]{babel}
\usepackage{tabularx}
\usepackage{enumerate}

\usepackage[table]{xcolor}

% Set column space
\setlength{\columnsep}{0.25em}

% Define colours
\definecolorset{hsb}{}{}{red,0,.4,0.95;orange,.1,.4,0.95;green,.25,.4,0.95;yellow,.15,.4,0.95}
\definecolorset{hsb}{}{}{blue,.55,.4,0.95;purple,.7,.4,0.95;pink,.8,.4,0.95;blue2,.58,.4,0.95}
\definecolorset{hsb}{}{}{magenta,.9,.4,0.95;green2,.29,.4,0.95}

% Redefine sections
\makeatletter
\renewcommand{\section}{\@startsection{section}{1}{0mm}
	{-1.7ex}{0.7ex}{\normalfont\large\bfseries}}
\renewcommand{\subsection}{\@startsection{subsection}{2}{0mm}
	{-1.7ex}{0.5ex}{\normalfont\normalsize\bfseries}}
\makeatother

% No section numbers
\setcounter{secnumdepth}{0}

% No indentation
\setlength{\parindent}{0em}

% No header and footer
\pagestyle{empty}


% A few shortcuts
\newcommand{\cmd}[1] {\texttt{\textbf{{#1}}}}
\newcommand{\cmdline}[1] {
	\begin{tabularx}{\hsize}{X}
			\texttt{\textbf{{#1}}}
	\end{tabularx}
}

\newcommand{\colouredbox}[2] {
	\colorbox{#1!40}{
		\begin{minipage}{0.95\linewidth}
			{
			\rowcolors[]{1}{#1!20}{#1!10}
			#2
			}
		\end{minipage}
	}
}

% redefine enumerate env for closer spacing
\renewenvironment{enumerate}%
  {\begin{list}{\arabic{enumi}.}%
     {\topsep=0in\itemsep=0in\parsep=0in\partopsep=0in\usecounter{enumi}}%
   }{\end{list}}


\begin{document}

\centerline{\Large{\textbf{Git Cheat Sheet}}}
%% \centerline{{\large{\textbf{\textit{git: first it sucks, then it's awesome}}}}}

\colouredbox{green}{
\section{Documentation \textmd{\textrm{(http://progit.org/book/)}}}
\cmdline{git help [cmd]}
}

\colouredbox{magenta}{
\section{For SVN Users}

\subsection{Tracking Projects}
\begin{tabularx}{\hsize}{lX}
\cmd{git clone url} & \cmd{svn checkout url} \\
\cmd{git pull} & \cmd{svn update} \\
\end{tabularx}

\subsection{Create Repository}
\begin{tabularx}{\hsize}{lX}
\cmd{git init} & \cmd{svnadmin create repo} \\
\cmd{git add .} & \cmd{svn import file://repo} \\
\cmd{git commit} &  \\
\end{tabularx}

\subsection{Files}
\begin{tabularx}{\hsize}{lX}
\cmd{git add file} & \cmd{svn add file} \\
\cmd{git rm file} & \cmd{svn rm file} \\
\cmd{git mv file} & \cmd{svn mv file} \\
\end{tabularx}

\subsection{Committing}
\begin{tabularx}{\hsize}{lX}
\cmd{git commit -a } & \cmd{svn commit} \\
\end{tabularx}

\subsection{Browsing}
\begin{tabularx}{\hsize}{lX}
\cmd{git log} & \cmd{svn log | less} \\
\cmd{git blame file} & \cmd{svn blame file} \\
\cmd{git diff} & \cmd{svn diff | less} \\
\end{tabularx}

}


\colouredbox{blue}{
\section{Getting Started}

\subsection{Setup}
\begin{tabularx}{\hsize}{lX}
Install:& \cmd{sudo port install git-core +svn}\\
Name:   & \cmd{git config --global user.name \textbackslash}\\
        & \cmd{  "Nate Murray"}\\
Email:  & \cmd{git config --global user.email \textbackslash}\\
        & \cmd{  "nate@xcombinator.com"}\\
\end{tabularx}

\subsection{First Repo}

\begin{tabularx}{\hsize}{lX}
 & \cmd{cd $\sim$} \\
 & \cmd{mkdir -p projects/demo} \\
 & \cmd{cd projects/demo} \\
 & \cmd{git init} \\
 & \cmd{git status} \\
 & \cmd{ls -a} \\
 & \cmd{echo "version 1" > test.txt} \\
 & \cmd{git status} \\
 & \cmd{git add test.txt} \\
 & \cmd{git status} \\
 & \cmd{git commit -m "added version one"} \\
 & \cmd{git status} \\

\end{tabularx}

\subsection{Aliases}
\begin{tabularx}{\hsize}{lX}
st:&\cmd{git config --global alias.st status} \\
ci:&\cmd{git config --global alias.ci commit} \\
ca:&\cmd{git config --global alias.ca commit -a} \\
\end{tabularx}
}


\colouredbox{green}{
\section{GUI}
\begin{tabularx}{\hsize}{lX}
Commits: & \cmd{gitx} \\
Merge: & \cmd{git mergetool -t opendiff} \\
\end{tabularx}

}

\colouredbox{orange}{
\section{Internals}

\begin{tabularx}{\hsize}{lX}
View Log: & \cmd{git log} \\
Show: & \cmd{git show a8dn3  \# sha}  \\
Cat: & \cmd{git cat-file -p 9s2j \# sha} \\
\end{tabularx}

}

\colouredbox{yellow}{
\section{Branching \& Merging}
\begin{tabularx}{\hsize}{lX}
List:     & \cmd{git branch} \\
Create:   & \cmd{git branch testing} \\
Checkout: & \cmd{git checkout testing} \\
          & \cmd{\# work, make some commits} \\
Back:     & \cmd{git checkout master} \\
Merge:    & \cmd{git merge testing} \\
Create:   & \cmd{git checkout -b new-branch} \\
Delete:   & \cmd{git branch -d new-branch} \\
\end{tabularx}
}





\colouredbox{blue2}{
\section{Managing Remotes}
\begin{tabularx}{\hsize}{lX}
Clone: &     \cmd{git clone url} \\
List: &     \cmd{git remote -v} \\
Send: &     \cmd{git push } \\
      &     \cmd{git push origin master} \\
Recv: &     \cmd{git pull } \\
      &     \cmd{git pull origin master} \\
Add:  &     \cmd{git remote add name url} \\
Fetch:&     \cmd{git fetch name} \\
Checkout:&  \cmd{git checkout name/master} \\
\end{tabularx}
}

\colouredbox{red}{
\section{3 Principles}
\begin{enumerate}{}{}
  \item (Almost) everything is local
  \item Git commits are snapshots
  \item Branching is cheap, use it often
\end{enumerate}
}

\colouredbox{yellow}{
\section{Next Steps}
\begin{tabularx}{\hsize}{lX}
  & \cmd{git tag name} \\
  & \cmd{git rebase commit} \\
  & \cmd{git cherry-pick commit} \\
  & \cmd{git bisect} \\
  & \cmd{git-svn} \\
  & hooks \\
  & tracking branches \\
  & submodules \\
  & interactive staging \\
  & squashing commits \\
  & repository hosting \\
  & patches via email \\
  & \cmd{gitjour} \\
\end{tabularx}
}






\begin{flushright}
\footnotesize
\rule{0.7\linewidth}{0.25pt}
\verb!$Revision: 1.1, $Date: August 31, 2010!\\
\verb!Nate Murray (nmurray@xcombinator.com)!
\verb!See also: http://book.git-scm.com/!
\verb!          http://git.or.cz/course/svn.html!
\verb!Based on Steve Taylon's Clojure Cheat Sheet!
\end{flushright}

% history
%
% v1.00
% First version based on API documentation
% http://incanter.org/docs/api
% LaTeX format based on Steve Tayon's Clojure Cheat Sheet
%
% v1.1
% Removed redundancies from Charts and Plots section

\end{document}

